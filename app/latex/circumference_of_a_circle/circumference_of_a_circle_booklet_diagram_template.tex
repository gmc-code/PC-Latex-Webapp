\begin{minipage}{0.55\textwidth}
  \refstepcounter{minipagecount}
  \noindent{(\theminipagecount)}\quad
  \begin{tikzpicture}[scale=1.0, baseline=(current bounding box.north)]
    \begin{scope}[rotate=0]
        \coordinate (A) at (0,0);
        % Define B using polar coordinates from A
        \coordinate (B) at ($(A) + (<<rotation>>:<<draw_radius>>)$);
        \fill (A) circle(1pt);
        \draw[thick] (A) circle (<<draw_radius>>);
        \draw[<->, gray, shorten <=2pt, shorten >=1.5pt]
          (A) -- (B)
          node[pos=0.35, sloped, above, fill=white, inner sep=2pt, xshift=0pt, yshift=3pt, transform shape]
          {\textcolor{black}{$<<calc_radius>>\,\text{<<units>>}$}};
    \end{scope}
  \end{tikzpicture}
\end{minipage}%
\hfill
\begin{minipage}{0.45\textwidth}
  \begin{align*}
  \text{Circumference} &= 2\pi r \\
  \text{Circumference} &= <<calc_formula_part1>> \times <<calc_formula_part2>> \times <<calc_radius_value>>\,\text{<<calc_units>>} \\
  \text{Circumference} &\approx <<calc_circumference_value>> \,\text{<<calc_units>>}
  \end{align*}
\end{minipage}