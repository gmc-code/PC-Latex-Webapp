
\begin{equation}
  \text{<<angle_to_find>>} = \text{<<angle_to_find_value>>}^\circ
  % scale to 0.85 to consistently fit 4 per page
  \begin{tikzpicture}[scale=0.85, baseline=(current bounding box.north)]
    \begin{scope}[rotate=<<rotationAngle>>]
      % Draw the first line
      \draw[<->>, >={Stealth[scale=1.3]}, name path=P1] (<<parIstartx>>, <<parIstarty>>) -- (<<parIendx>>, <<parIendy>>);
      % Draw the second line with the calculated offsets
      \draw[<->>, >={Stealth[scale=1.3]}, name path=P2] (<<parIIstartx>>, <<parIIstarty>>) -- (<<parIIendx>>, <<parIIendy>>);
      % Draw the transversal through the middle of the parallel lines
      \draw[<->, >=Stealth, name path=P3] (<<transstartx>>, <<transstarty>>) -- (<<transendx>>, <<transendy>>);
      \path [name intersections={of=P1 and P3,by=A}];
      \path [name intersections={of=P2 and P3,by=B}];
      % Draw the angle
      \coordinate (p1s) at (<<parIstartx>>, <<parIstarty>>);
      \coordinate (p1e) at (<<parIendx>>, <<parIendy>>);
      \coordinate (p2s) at (<<parIIstartx>>, <<parIIstarty>>);
      \coordinate (p2e) at (<<parIIendx>>, <<parIIendy>>);
      \coordinate (ts) at (<<transstartx>>, <<transstarty>>);
      \coordinate (te) at (<<transendx>>, <<transendy>>);
      <<anglestext>>
    \end{scope}
  \end{tikzpicture}
\end{equation}